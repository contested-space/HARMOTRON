\documentclass{article}
\usepackage[francais]{babel}
\usepackage[utf8]{inputenc}
\usepackage[T1]{fontenc}
\usepackage{tipa}
\usepackage{amsmath, amssymb}
\usepackage{hyperref}

\setlength\parindent{0in}
\setlength\topmargin{-0.25in}
\setlength\headheight{0in}
\setlength\headsep{0in}
\setlength\topskip{0.5in}
\setlength\textheight{9.75in}
\setlength\textwidth{6.50in}
\setlength\oddsidemargin{0in}
\setlength\evensidemargin{0in}

\title{HARMOTRON}
\author{F}
\date{\today}

\begin{document}

\section{Concept}
The Harmotron is an idea for a dual control synthesizer. One of the controls
will be the pitch of the tone generator (a harmonic rich sound) while the other
will control the bandpass frequencies. This way, the user will be able play
on both the main note (background) and the isolated frequencies for harmonic
effect.

\subsection{todo}
\begin{description}
\item continue to test effect of a specific bandpass fc on static signals
\item find appropriate serie with shiftable parameter (like exponent exp<->log)
\item generate sines on user input (midi input???). 
\item generate appropriate filter on user input.
\item implement adsr
\end{description}


\section{Source synthesis}
The source will be synthesized by adding sinusoids that are integer multiples of
the fundamental frequency (user input). 

Each harmonic will have an amplitude equal to its term in a serie determined
by user input (harmonic, geometric, or something like that).

\subsection{Formulaes and results}
The following synthesis formulaes have been used and tested:
\begin{description}
\item 1)$synth(f, t) = \sum\limits_{n=1}^N a^n cos(2 \pi f t n /F_s)$
\item 2)$synth(f, t) = \sum\limits_{n=1}^N a^n cos(2 \pi f t n /F_s)
                     + \sum\limits_{n=1}^N a^n cos(2 \pi f t n /F_s * \frac{3}{2})
                     + \sum\limits_{n=1}^N a^n cos(2 \pi f t n /F_s * 3)$
\end{description}

It could be interesting to generate different type of spacing for the harmonics.
Different series of ratios of other enharmonic (or is it xenharmonic?) ideas.


\section{filtering}
The filter will be a bandpass (maybe morphing to a comb filter centered on $F_c$).
$F_c$ will be dynamically determined by user input. 

\subsection{reflexions on filtering}
After a few initial tests, the concept works but needs refining. Some amplitudes
were too low, others were a bit dissonant. The present hypothesis is that the
filter frequency $F_c$ does not perfectly align with all harmonics, causing loss
of amplitude and dissonances from letting pass untargeted frequencies in a 
greater ratio. 

As it is, a single harmonic may be passed at a time. By using parrallel filters, 
it would be possible to play simultaneous harmonics by computing them all and
controlling their amplitudes by user input (i.e. on/off from keyboard input).

\section{envelope}
\subsection{initial gain}
Since the signal is the result of coefficients of a geometric series multiplied
by the amplitude of the \texttt{cos} function, it can be multiplied by its 
inverse to reduce peaking. The total sum of a geometric series is known to be
$\sum\limits_{n=1}^N a^n = \frac{1}{1 - a}$ if $\left|a\right| < 1$. It's
iverse is $1 - a$. We estimate the energy of the \texttt{cos} function by
taking $\int_{\frac{-\pi}{2}}^{\frac{\pi}{2}} \! cos(x)\, \mathrm{d}x = 2$ and 
its inverse is of course $\frac{1}{2}$. Since int the case of the second
synthesis formula, the series is applied three times,it seems wise to divide 
the gain by 3.

So the initial gain will be $\frac{1-a}{6}$

\end{document}